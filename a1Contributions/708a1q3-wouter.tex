\documentclass{article}
\usepackage[utf8]{inputenc}
\usepackage{amsmath, amssymb}
\usepackage{graphicx}
\newcommand{\calG}{\mathcal{G}}
\title{Assignment 1 Question 3}
\author{Wouter van Zeist}
\date{}
\begin{document}
	\maketitle

The equations of motion generated by the Hamiltonian give us the following:
\begin{align*}
&\frac{dq_{i}}{dt}=\frac{\partial H}{\partial p_{i}}\\
&\frac{dp_{i}}{dt}=-\frac{\partial H}{\partial q_{i}}
\end{align*}

Therefore, we can reduce the following equation required for the conservation of the Hamiltonian:
\begin{align*}
&\frac{dH}{dt}=\frac{\partial H}{\partial p}\frac{\partial p}{\partial t}+\frac{\partial H}{\partial q}\frac{\partial q}{\partial t}+\frac{\partial H}{\partial t}\\
&=\frac{dq}{dt}\frac{\partial p}{\partial t}-\frac{dp}{dt}\frac{\partial q}{\partial t}+\frac{\partial H}{\partial t}\\
&=\frac{d^{2}pq}{dt}-\frac{d^{2}pq}{dt}+\frac{\partial H}{\partial t}\\
&=\frac{\partial H}{\partial t}\\
&=0
\end{align*}

This gives us $\frac{dH}{dt}=0$, i.e. the Hamiltonian remains constant over time.\\

The fact that the Hamiltonian remains constant over time for valid solutions of the equations of motions means that, firstly, the solutions of the Hamiltonian conserve the Hamiltonian and therefore the total energy. This is just another statement of the principle of conservation of energy, and shows that these Hamiltonian systems obey this part of classical mechanics.

Furthermore, knowing that $\frac{dH}{dt}=0$ removes one dimension of freedom from the phase space. The entropy of a system depends on the accessible phase space volume of the system by the fundamental postulate of thermodynamics, $S=k_{B}\ln(\Omega)$, where $k_{B}$ is Boltzmann's constant and $\Omega$ is the accessible phase space volume. This volume obviously decreases as a dimension is removed, which means that the entropy of the system if the Hamiltonian remains constant over time would be less than if it did not.

\end{document}